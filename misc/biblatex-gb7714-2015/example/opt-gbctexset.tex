
% !Mode:: "TeX:UTF-8"
% 用于测试gb7714-2015样式,是否可以利用ctex的设置对文献表的标题进行修改
% 测试gbctexset选项
\documentclass[twoside]{article}
\usepackage{ctex}
\usepackage[top=10pt,bottom=10pt,left=1cm,right=1cm]{geometry}
\usepackage{xcolor}
\usepackage[CJKbookmarks,colorlinks,bookmarksnumbered=true,pdfstartview=FitH,linkcolor=blue]{hyperref}
\usepackage[backend=biber,style=gb7714-2015ay,gbctexset=true]{biblatex}
\addbibresource[location=local]{example.bib}

\setlength{\bibitemsep}{1pt}%

%gbctexset=true,时可利用ctexset来设置文献表标题
\ctexset{
    bibname={ctexset设置的中文参考文献}
}

%gbctexset=false,时可利用本地化字符串来设置文献表标题
\DefineBibliographyStrings{english}{
bibliography={本地化字符串设置的ref title},
references={本地化字符串设置的ref title},
}


\iftoggle{iftleight}{\defdoublelangentry{易仕和2013--}{Yi2013--}}{}
\iftoggle{iftlnine}{\defdoublelangentry{易仕和2013--}{Yi2013--}}{}
\iftoggle{iftlatest}{\defdoublelangentry{易仕和2013--}{Yi2013--}}{}

\begin{document}

\section{测试利用ctexset或本地化字符串设置文献表标题}

\begin{refsection}
\defbibentryset{张敏莉,等,2007}{张敏莉2007-500-503,Zhang2007-500-503}
双语文献:set动态方法\cite{张敏莉,等,2007};
\iftoggle{iftleight}{related动态方法\cite{易仕和2013--}}{}
\iftoggle{iftlnine}{related动态方法\cite{易仕和2013--}}{}
\iftoggle{iftlatest}{related动态方法\cite{易仕和2013--}}{}

文献\cite{王夫之1845--}\cite{陈建军2010-93-93};

\printbibliography[heading=subbibintoc]
\end{refsection}

\section{测试利用printbibliography的title选项设置文献表标题}

\begin{refsection}
文献\parencite{张田勤2000--}\parencite{吴云芳2003--};
文献\pagescite[][300]{汤万金2013-09-30--}\pagescite[][100-107]{张凯军2012-04-05--};
文献:萧钰\yearpagescite{萧钰2001--},
国家环境保护局科技标准司\yearpagescite{国家环境保护局科技标准司1996-2-3};
其它\cite{Calkin2011-8-9,CRAWFPRD1995--,
Babu2014--,CALMS1965--,DESMARAIS1992-605-609}

\printbibliography[heading=subbibintoc,title=References]
\end{refsection}
\end{document} 