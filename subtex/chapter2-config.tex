% !Mode:: "TeX:UTF-8"
\ifdefined\maindoc
\else
  \documentclass[master,openright,twoside,color,AutoFakeBold=true]{misc/buaathesis}
  \addbibresource{reference_thesis.bib}
  \begin{document}
  % 正文页码样式
  \mainmatter
  % 正文页眉页脚样式 
  \pagestyle{mainmatter}
\fi

\chapter{环境配置}

\section{C\TeX{}套装 [Windows Only]}

C\TeX{}套装是 Windows 下为中文优化的\LaTeX{}系统套件,主要基于 MiKTeX 系统,
集成了编辑器 WinEdt 和其他相关软件。整个系统封装在一个安装程序中,
安装方法与常规软件相同,无需任何配置,适合大部分 Windows 用户使用。

\begin{description}
    \item[下载地址] \hfill
    \begin{description}
        \item[官方页面]
            \url{http://www.ctex.org/CTeXDownload}
        \item[清华镜像]
            \url{https://mirrors.tuna.tsinghua.edu.cn/ctex/unstable/}
        \item[中科大镜像]
            \url{http://mirrors.ustc.edu.cn/ctex/unstable/}
    \end{description}
    \item[安装方法] \hfill
        \begin{itemize}
            \item[] 与常规软件的安装方法类似
            \item[] 一直下一步稍加一些自定义(如安装路径)即可
            \item[] {\bf 注意:} 安装程序在某些情况下可能覆盖 PATH 环境变量,请在安装前注意备份 PATH 环境变量
        \end{itemize}
\end{description}

\section{\TeX{}Live [ Windows \& Linux ]}

\TeX{}是自由软件,有很多发行版本,就像 Linux 的 Ubuntu、Fedora 等等。
每个发行版本都是一套工具集合,包括 plain\TeX{},\LaTeX{},pdf\TeX{},dvips 等。
其中比较流行的是\TeX{}Live,也包含在 CTAN 的开源镜像中。

推荐通过下载 ISO 镜像文件的方式安装:
\begin{description}
    \item[官方说明]
        \url{http://www.tug.org/texlive/acquire-iso.html}
    \item[下载地址] 官方地址会自动跳转寻找"最近"镜像,还有几个较快的教育网镜像
    \begin{description}
        \item[官方地址]
            \url{http://mirror.ctan.org/systems/texlive/Images/texlive2016.iso}
        \item[清华镜像]
            \url{http://mirrors.tuna.tsinghua.edu.cn/CTAN/systems/texlive/Images/}
        \item[中科大镜像]
            \url{https://mirrors.ustc.edu.cn/CTAN/systems/texlive/Images/}
    \end{description}
    \item[安装方法] \hfill
    \begin{enumerate}
        \item 通过虚拟光驱挂载镜像也可以直接打开或解压缩不过会比较慢
        \item 双击运行光盘镜像或者运行脚本
        \item[] Windows 用户可以直接双击运行\textsl{install-tl.bat}
        \item[] Linux 用户可以在终端下执行命令\textsl{./install-tl}
        \item 按照提示下一步即可,安装大致耗时 10$\sim$20 分钟,受机器配置影响。
    \end{enumerate}
\end{description}

当然官方也提供了通过网络安装的方式,虽然通过可以通过镜像选择达到比较快的速度,
但是这里简便期间不再赘述,有兴趣的同学可以参考官方说明
\url{http://www.tug.org/texlive/acquire-netinstall.html}。

\section{Mac\TeX{} [ Mac ]}

Mac\TeX{}是基于\TeX{}Live 为 Mac 系统设计的套件。

\begin{description}
    \item[官方网站]
        \url{http://tug.org/mactex/}
    \item[下载地址] 官方地址会自动跳转寻找"最近"镜像,还有几个较快的教育网镜像
    \begin{description}
        \item[官方地址]
            \url{http://mirror.ctan.org/systems/mac/mactex/MacTeX.pkg}
        \item[清华镜像]
            \url{http://mirrors.tuna.tsinghua.edu.cn/CTAN/systems/mac/mactex/}
        \item[中科大镜像]
            \url{https://mirrors.ustc.edu.cn/CTAN/systems/mac/mactex/}
    \end{description}
    \item[安装方法] 同一般软件安装,下一步即可
\end{description}

\section{兼容性}

本模板依赖 v2.0 及以上版本的 ctex 包,\TeX{}Live 2015 及以上版本、C\TeX{}2.9.3 可以正常使用。
对于低版本的\LaTeX{}发行版,需要使用包管理器升级 ctex 宏包。

\section{安装字体 [ Linux ]}

北航的毕业设计论文要求使用 Times New Roman 和华文行楷这两种字体,在 Linux 系统上,这两种字
是没有预装在系统里的,因此 Linux 用户需要手动安装字体才能正常使用本模板。本节将以 Ubuntu 系统
为例演示字体的安装。

首先需要获取字体文件,Windows 系统默认包含了 Times New Roman 和华文行楷这两种字体,可以从
\verb|C:\Windows\Fonts\|文件夹下将字体文件拷贝出来(显示为\verb|STXingkai|和
\verb|Times New Roman|),当然,用户也可以从其他途径获取这两个字体文件。然后将字体文件
拷贝到 Ubuntu 的\verb|/usr/share/fonts|目录下,为了方便管理,可以在这些外部字体放在一个新
文件夹中:
\begin{lstlisting}[
    language={bash},
    caption={拷贝字体文件},
    label={copy-fonts},
]
sudo cp <your font files> /usr/share/fonts/msfonts/
\end{lstlisting}
然后将字体文件的权限设置为 644:
\begin{lstlisting}[
    language={bash},
    caption={设置字体文件权限},
    label={set-fonts-permission},
]
sudo chmod 644 /usr/share/fonts/msfonts/*
\end{lstlisting}
接下来,进入到{\verb /usr/share/fonts/msfonts } 目录下,依次运行以下三个命令:
\begin{lstlisting}[
    language={bash},
    caption={安装字体},
    label={install-fonts},
]
sudo mkfontscale
sudo mkfontdir
sudo fc-cache -fv
\end{lstlisting}
当看到命令行输出
\begin{lstlisting}[
    language={bash},
    caption={正常输出结果},
    label={install-font-success},
]
fc-cache: succeeded
\end{lstlisting}
时,就完成了字体的安装。

\section{关于编辑器}

以上介绍了三款\LaTeX{}套装,涵盖了主流的三大平台,除了 C\TeX{}自带了 WinEdt,
其余两款均需要自己选择编辑器,理论上任何文本编辑器都是可以使用的,
如 Windows 上的 vscode,Linux/MacOS 上的 vim,emacs,
一方面要考虑对\LaTeX{}的支持,一方面还是自己的熟悉程度。

这里推荐一款大众化的编辑器\TeX{}maker,它是跨平台的,支持 Windows、Linux 和 MacOS。

\begin{description}
    \item[官方网站]
        \url{http://www.xm1math.net/texmaker/}
    \item[下载地址]
        \url{http://www.xm1math.net/texmaker/download.html}
    \item[相关说明]
    \begin{itemize}
        \item 安装同一般软件的安装
        \item 配置 Xe\LaTeX{}的编译,选择菜单栏“选项”->“配置\TeX{}Maker”,
        \item[] 在“\LaTeX{}”一栏填写
            \texttt{xelatex -interaction=nonstopmode\%.tex}
    \end{itemize}
\end{description}

\section{关于编译}

\LaTeX{}的文件是通过编译生成的,对于本模板和毕业设计论文而言,
需要经过代码\ref{code-compile}所示步骤(以 sample-bachelor.tex 为例):
\begin{lstlisting}[
    language={bash},
    caption={编译步骤},
    label={code-compile},
]
xelatex sample-bachelor.tex
bibtex  sample-bachelor.aux
xelatex sample-bachelor.tex
xelatex sample-bachelor.tex
\end{lstlisting}
当然,我们在模板里也提供了编译的执行脚本。

\subsection{批处理 [ Windows only ]}

进入 cmd(Win+R,然后输入 cmd),cd 到 BUAAthesis 对应目录,
如\verb|D:\BUAAthesis\|,然后运行\verb|msmake|即可。

\subsection{Makefile [ Windows(Cygwin) / Linux / MacOS ]}
需要要你的命令行环境支持 Make,cd 到 BUAAthesis 相应目录,
目前支持如代码\ref{code-make}所示的功能:
\begin{lstlisting}[
    language={bash},
    caption={make 命令},
    label={code-make},
]
make bachelor # 编译本科生的\LaTeX{}(文件默认项,亦可直接输入 make)
make master # 编译研究生的\LaTeX{}文件
make kaitireport # 编译本科生/研究生的开题报告/文献综述的\LaTeX{}文件
make clean # 删除编译过程中生成的文件(除了 pdf)
make depclean # 删除编译过程中生成的文件(包括 pdf)
\end{lstlisting}

\ifdefined\maindoc
\else
  % !Mode:: "TeX:UTF-8"
% https://blog.csdn.net/X_And_Y/article/details/104867559
% 实际发现出来的作者名都大写了,这是因为bibtex样式文件的标准规定如此 
% 解决方法:修改biblatex的gbnamefmt=lowercase参数 https://github.com/sikouhjw/gdutthesis/discussions/66
\cleardoublepage
\phantomsection
\addcontentsline{toc}{chapter}{参考文献}
% \nocite{*}
\printbibliography[title=参\ 考\ 文\ 献] % biblatex 方式
\cleardoublepage
  \end{document}
\fi