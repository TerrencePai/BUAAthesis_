% !Mode:: "TeX:UTF-8"
\ifdefined\maindoc
\else
  \documentclass[master,openright,twoside,color,AutoFakeBold=true]{misc/buaathesis}
  \begin{document}
\fi


\frontmatter
% 前言页眉页脚样式
\pagestyle{frontmatter}

% 中英文摘要
\begin{cabstract}
  中文摘要内容包括:“摘要”字样,摘要正文,关键词。对于中英文摘要,都必须在摘要的最下方另起一行,用显著的字符注明本文的关键词,格式见附件13。
  摘要是学位论文内容的简短陈述,应体现论文工作的核心思想。论文摘要应力求语言精练准确,博士学位论文的中文摘要一般约800~1200字;硕士学位论文的中文摘要一般约500字。摘要内容应涉及本项科研工作的目的和意义、研究思想和方法、研究成果和结论,博士学位论文必须突出论文的创造性成果,硕士学位论文必须突出论文的新见解。
  关键词:是为用户查找文献,从文中选取出来用来揭示全文主题内容的一组词语或术语,应尽量采用词表中的规范词(参照相应的技术术语标准)。关键词一般为3~5个,按词条的外延层次排列(外延大的排在前面)。关键词之间用逗号分开,最后一个关键词后不打标点符号。  
\end{cabstract}

\begin{eabstract}
  英文摘要:为了国际交流的需要,论文须有英文摘要。英文摘要的内容及关键词应与中文摘要及关键词一致,要符合英语语法,语句通顺,文字流畅。英文和汉语拼音一律为Times New Roman体,字号与中文摘要相同,格式见附件14。
\end{eabstract}

\ifdefined\maindoc
\else
  \end{document}
\fi