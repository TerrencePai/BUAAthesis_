% !Mode:: "TeX:UTF-8"

% 学院中英文名,中文不需要“学院”二字
% 院系英文名可从以下导航页面进入各个学院的主页查看
% https://www.buaa.edu.cn/jgsz/yxsz.htm
\school
{电子信息工程}{School of Electronics and Information Engineering}

% 专业中英文名
\major
{信息与通信工程}{Information and Communication Engineering}

% 论文中英文标题
\thesistitle
{中文标题}
{中文标题} %(题名页标题) 一般情况直接复制一次,当换行位置不一样时可以在此处修改
{} %(副标题)
{Ying Wen Biao Ti}
{} %(Subtitle)

% 作者中英文名
\thesisauthor
{佚名}{Yi Ming}

% 导师中英文名
\teacher
{佚名}{Yi Ming}
% 副导师中英文名
% 注:慎用‘副导师’,见北航研究生毕业论文规范
%\subteacher{(副导师姓名)}{(Name of Subteacher)}

% 中图分类号,可在 http://www.ztflh.com/ 查询
\category{TN95}

% 本科生为毕设开始时间;研究生为学习开始时间
\thesisbegin{2021}{9}{1}

% 本科生为毕设结束时间;研究生为学习结束时间
\thesisend{2024}{6}{1}

% 毕设答辩时间
\defense{(年)}{(月)}{(日)}

% 声明页和授权书页时间
\creationandusedate{(年)}{(月)}{(日)}

% 中文摘要关键字
\ckeyword{多传感器,图像配准,图像融合,SAR图像,光学图像}

% 英文摘要关键字
\ekeyword{Multi-Sensor, Image Fusion, Fusion Effect, SAR image, Optical Image}
